\documentclass[12pt]{article}
\usepackage[utf8]{inputenc}
\usepackage[T1]{fontenc}
\usepackage[turkish]{babel}
\usepackage{graphicx}
\usepackage{listings}
\usepackage{xcolor}
\usepackage{amsmath}
\usepackage{hyperref}
\usepackage{algorithm}
\usepackage{algorithmic}

% Code listing settings
\lstset{
    language=Python,
    basicstyle=\ttfamily\small,
    keywordstyle=\color{blue},
    stringstyle=\color{red},
    commentstyle=\color{green!60!black},
    numbers=left,
    numberstyle=\tiny,
    numbersep=5pt,
    frame=single,
    breaklines=true,
    breakatwhitespace=true,
    showstringspaces=false,
}

\title{Bilgisayar Uygulamalarında Çizge Kuramı Projesi}
\author{Emre Yıldız}
\date{\today}

\begin{document}

\maketitle

\begin{abstract}
Bu projede, çizge kuramının temel kavramlarını Python programlama dili kullanarak uyguladık. Proje, çizgelerin oluşturulması, analizi ve manipülasyonu için gerekli temel işlevleri içermektedir. Özellikle, tam çizge, döngü çizge, çark çizge ve n-küp çizge gibi özel çizge türlerinin oluşturulması ve analizi için gerekli algoritmalar implement edilmiştir. Projede herhangi bir harici kütüphane kullanılmamıştır. Tüm algoritmalar, veri yapıları ve sınıflar sıfırdan geliştirilmiştir.
\end{abstract}

\tableofcontents

\section{Giriş}
Çizge kuramı, matematiğin önemli alanlarından biridir ve bilgisayar bilimlerinde geniş uygulama alanına sahiptir. Bu projede, çizge kuramının temel kavramlarını Python programlama dilinde uygulayarak, çizgelerin oluşturulması, analizi ve manipülasyonu için bir yazılım kütüphanesi geliştirdik.

\section{Tasarım ve Implementasyon}

\subsection{Veri Yapıları}
Projede iki temel veri yapısı kullanılmıştır:

\subsubsection{Tepe (Vertex) Sınıfı}
Tepe sınıfı, çizgedeki her bir düğümü temsil eder:
\begin{itemize}
    \item \texttt{label}: Tepenin etiketi (string)
    \item \texttt{neighbors}: Komşu tepelerin kümesi (Set[str])
\end{itemize}

Python'un \texttt{dataclass} özelliği kullanılarak implement edilmiştir. Bu sayede sınıf otomatik olarak \texttt{\_\_init\_\_}, \texttt{\_\_repr\_\_} ve \texttt{\_\_eq\_\_} metodlarına sahip olur.

\subsubsection{Çizge (Graph) Sınıfı}
Ana çizge sınıfı şu özelliklere sahiptir:
\begin{itemize}
    \item \texttt{vertices}: Tepeler sözlüğü (Dict[str, Vertex])
    \item \texttt{input\_type}: Giriş formatı (Matrix/List)
\end{itemize}

\section{Algoritmalar ve Analiz Metodları}

\subsection{Çizge Türü Belirleme}
\texttt{determine\_graph\_type} metodu, bir çizgenin özel türlerden biri olup olmadığını kontrol eder:

\begin{algorithm}
\caption{Çizge Türü Belirleme}
\begin{algorithmic}
\STATE \textbf{Input:} Graph G
\STATE \textbf{Output:} (is\_complete, is\_cycle, is\_wheel, is\_hypercube)
\STATE
\STATE // Tam çizge kontrolü
\IF{her tepenin derecesi (n-1) ise}
    \STATE is\_complete = true
\ENDIF
\STATE
\STATE // Döngü çizge kontrolü
\IF{n $\geq$ 3 ve her tepenin derecesi 2 ve çizge bağlı ise}
    \STATE is\_cycle = true
\ENDIF
\STATE
\STATE // Çark çizge kontrolü
\IF{n $\geq$ 4}
    \STATE merkez = derece(n-1) olan tepe
    \IF{merkez varsa ve kalan tepeler döngü oluşturuyorsa}
        \STATE is\_wheel = true
    \ENDIF
\ENDIF
\STATE
\STATE // n-küp çizge kontrolü
\IF{n = 2\^{}k ve her tepenin derecesi k ise}
    \STATE is\_hypercube = true
\ENDIF
\end{algorithmic}
\end{algorithm}

\subsection{İki Parçalı Çizge Kontrolü}
İki parçalı çizge kontrolü için BFS tabanlı bir renklendirme algoritması kullanılmıştır:

\begin{algorithm}
\caption{İki Parçalı Çizge Kontrolü}
\begin{algorithmic}
\STATE \textbf{Input:} Graph G
\STATE \textbf{Output:} (is\_bipartite, (set1, set2))
\STATE
\STATE colors = \{\} // Tepe → renk (0 veya 1)
\FOR{her başlangıç tepesi v}
    \IF{v renklendirilmemişse}
        \STATE queue = [(v, 0)]
        \WHILE{queue boş değil}
            \STATE vertex, color = queue.pop()
            \IF{vertex renklendirilmişse}
                \IF{rengi farklıysa}
                    \RETURN (false, None)
                \ENDIF
                \STATE continue
            \ENDIF
            \STATE vertex'i color ile renklendir
            \STATE komşuları karşıt renkle queue'ya ekle
        \ENDWHILE
    \ENDIF
\ENDFOR
\STATE set0 = rengi 0 olan tepeler
\STATE set1 = rengi 1 olan tepeler
\RETURN (true, (set0, set1))
\end{algorithmic}
\end{algorithm}

\subsection{Bağlı Bileşenler}
Çizgenin bağlı bileşenlerini bulmak için BFS algoritması kullanılmıştır:

\begin{algorithm}
\caption{Bağlı Bileşenleri Bulma}
\begin{algorithmic}
\STATE \textbf{Input:} Graph G
\STATE \textbf{Output:} List[Set[str]] (bağlı bileşenler listesi)
\STATE
\STATE unvisited = tüm tepeler
\STATE components = []
\WHILE{unvisited boş değil}
    \STATE start = unvisited'dan bir tepe seç
    \STATE component = \{\}
    \STATE queue = [start]
    \WHILE{queue boş değil}
        \STATE vertex = queue.pop()
        \IF{vertex ziyaret edilmemişse}
            \STATE component'e ekle
            \STATE unvisited'dan çıkar
            \STATE ziyaret edilmemiş komşuları queue'ya ekle
        \ENDIF
    \ENDWHILE
    \STATE components'e component'i ekle
\ENDWHILE
\RETURN components
\end{algorithmic}
\end{algorithm}

\section{Performans Analizi}
Algoritmaların karmaşıklık analizi:

\begin{itemize}
    \item \textbf{Çizge Türü Belirleme}: O(V + E)
    \item \textbf{İki Parçalı Kontrol}: O(V + E)
    \item \textbf{Bağlı Bileşenler}: O(V + E)
    \item \textbf{Tam İki Parçalı Kontrol}: O(V\^{}2)
\end{itemize}

Burada V tepe sayısını, E kenar sayısını göstermektedir.

\section{Test ve Doğrulama}
Proje kapsamlı bir test paketi içermektedir. unittest modülü kullanılarak yazılan testler şunları içerir:

\begin{itemize}
    \item \textbf{Özel Çizge Testleri}:
    \begin{itemize}
        \item Tam çizge (Kn) oluşturma ve doğrulama
        \item Döngü çizge (Cn) oluşturma ve doğrulama
        \item Çark çizge (Wn) oluşturma ve doğrulama
        \item n-küp çizge (Qn) oluşturma ve doğrulama
    \end{itemize}
    \item \textbf{Çizge Özellikleri Testleri}:
    \begin{itemize}
        \item Soyutlanmış tepe bulma
        \item Asılı tepe bulma
        \item Bağlılık kontrolü
        \item İki parçalı çizge kontrolü
    \end{itemize}
    \item \textbf{Dosya İşlemleri Testleri}:
    \begin{itemize}
        \item Matris formatı okuma/yazma
        \item Liste formatı okuma/yazma
    \end{itemize}
\end{itemize}

\section{Sonuç}
Bu projede, çizge kuramının temel kavramları Python programlama dili kullanılarak başarıyla uygulanmıştır. Özellikle:

\begin{itemize}
    \item Farklı çizge türlerinin oluşturulması ve analizi
    \item Verimli veri yapıları ve algoritmalar
    \item Kapsamlı test ve doğrulama
    \item Detaylı dokümantasyon
\end{itemize}

başarıyla gerçekleştirilmiştir. Proje, çizge kuramı konusunda eğitim ve araştırma amaçlı kullanılabilir.

\appendix
\section{Kaynak Kodları}
Projenin tam kaynak kodları ayrı bir dokümanda (code\_appendix.tex) verilmiştir.

\end{document}
